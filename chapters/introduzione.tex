Negli ultimi anni, si è presentata una notevole evoluzione dei meccanismi principali sfruttati dai
creatori dei giochi da tavolo nell'ideazione di esperienze di gioco immersive, originali e stimolanti. \\
I nuovi generi sono caratterizzati da delle specifiche meccaniche e game flow (l'intero flusso di gioco):
gli \emph{engine building} come "Terraforming Mars" e "Gizmos", che spingono a sviluppare le interazioni 
tra i vari componenti acquisiti durante la partita per creare un motore con un solo obiettivo;
gli strategici "Scythe" o il classico "Scacchi", che si concentrano sull'abilità dei giocatori di pianificare 
e adattarsi ai cambiamenti dell'ambiente di gioco; i cooperativi "Pandemic Legacy" e "Gloomhaven", che uniscono i giocatori per adempiere a un 
singolo compito comune; i piazziamento lavoratori come "Feast for Odin" e "Dune Imperium", che obbligano a scegliere 
con accortezza le poche mosse e l'utilizzo delle scarse risorse a disposizione. \\
Gli avanzamenti nello sviluppo dei giochi da tavolo generano con sé delle difficoltà nel descrivere logicamente 
e in maniera strutturata i flussi di gioco, limitando la possibilità di effettuare la prototipizzazione 
direttamente in modo digitale delle meccaniche di possibili nuovi giochi da tavolo, 
prima ancora di creare una versione fisica. \\
Nelle seguenti pagine si illustrerà il principale linguaggio impiegato attualmente, 
seguito da una nuova alternativa per la descrizione dei giochi da tavolo. \\
L’obiettivo finale è sia di fornire un’alternativa ai metodi in uso a oggi, 
sia di stimolare e ispirare la discussione sull’argomento del general game playing. \\
Importante notare che la proposta non si prospetta come definizione rigida dei requisiti dell’implementazione 
del linguaggio, ma come linea guida comune di partenza per futuri sviluppi e
modifiche nella creazione di strumenti facilmente utilizzabili e apprendibili velocemente. 

