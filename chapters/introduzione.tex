Negli ultimi anni si è presentata una notevole evoluzione dei meccanismi principali sfruttati dai creatori
dei giochi da tavolo per creare esperienze immersive, originali e stimolanti  per i giocatori coinvolti. \\
I nuovi generi dei giochi si portano dietro le loro meccaniche caratteristiche,
come gli engine building con l’interazione tra componenti e la gestione delle risorse,
i giochi di strategia che si concentrano sull’abilità del giocatore di pianificare e adattarsi ai cambiamenti dell’ambiente di gioco, 
i cooperativi che unisco i giocatori per adempire a un singolo compito comune oppure i giochi di piazzamento lavoratori 
che obbligano a scegliere con accortezza le poche mosse a disposizione. \\
Gli avanzamenti nello sviluppo dei giochi da tavolo generano con sé delle difficoltà nel descrivere logicamente 
e in maniera strutturata i flussi di gioco, limitando la possibilità di effettuare la prototipizzazione 
direttamente in modo digitale delle meccaniche di possibili nuovi giochi da tavolo, prima ancora di creare una versione fisica. \\
Nelle seguenti pagine si illustrerà il principale linguaggio usato attualmente, 
seguito da una nuova alternativa per descrivere i giochi da tavolo, 
con l’obiettivo sia di fornire un’alternativa ai metodi in uso a oggi, 
sia per stimolare e inspirare la discussione sull’argomento del general game playing. \\
Importante notare che la proposta non si prospetta come definizione rigida dei requisiti dell’implementazione del linguaggio 
ma come linea guida comune di partenza e ispirazione per futuri sviluppi, 
aggiunte e modifiche per creare degli strumenti facilmente utilizzabili e apprendibili velocemente. \\

