\section{Nozioni Fondamentali}

\subsection{Identificatori}
Gli Identificatori sono una qualsiasi stringa di caratteri che inizia per un trattino basso "\_" o 
per lettera, opzionalmente seguiti da un qualsiasi ammontare di lettere o numeri senza spazi.

Un identificatore per essere legale deve essere unico all'interno del suo ambito di Visibilità,
questo significa che deve essere diverso sia dagli altri identificatori ma anche da operatori e clausole.  

\subsection{Operatore e Clausole}
Gli operatori e le clausole sono delle parole chiave o simboli riservati che permettono rispettivamente
la costruzione delle espressioni e delle istruzioni.

\subsection{Blocchi}
I Blocchi sono un insieme struttura d'istruzioni, tag e altri blocchi.
Un blocco si definisce come tutto il contenuto all'interno di una coppia di parentesi graffe,
rispettivamente aperta e chiusa. 
\\
Il blocco definire l'ambito, o scope, e vita di una variabile definita al suo interno ovvero
il suo campo d'azione che parte dall riga dov'è stata dichiarata fino alla fine del blocco 
che contiene la sua dichiarazione.
\\
La Visibilità di una variabile si estende a tutti gli ambiti figli del blocco di dichiarazione, 
dove figlio significa definito all'interno del padre (annidato).

 
\subsection{Tags}
I Tag sono delle parole chiave riservate che modificano il comportamento del blocco associato, 
ogni tag descrive cosa è permesso definire al suo interno.
All'interno della proposta è possibile che si trovi la dicitura "blocco <tag>" come abbreviazione
per indicare la coppia tag e relativo blocco associato.

 
\subsection{Commenti}
I commenti sono linee di codice che vengono ignorare durante l'esecuzione del programma, 
per commentare una riga è sufficiente inserire una doppia barra "//", qualsiasi scritta oltre 
le due barre è considerata parte del commento.

\section{Primitive del linguaggio}
Tabletop Game Description Language si basa su un insieme di primitive da cui sono costruire tutte 
le funzionalità del linguaggio.

 
\subsection{Oggetti}
Un oggetto è un'istanza di una certa classe, un blocco di memoria che viene allocato e configurato secondo
le specifiche definite dalla classe. \\
Gli oggetti hanno due comportamenti principali, gli oggetti valori e gli oggetti riferimento:
\\
Oggetti valore o Value objects, si riferiscono direttamente al valore, quando avviene una copia dell'oggetto
viene creata una nuova istanza della stessa tipologia di oggetto che contiene lo stesso valore.
Il comportamento degli oggetti valore è principalmente per i tipi primitivi come number, string, bool.
\\
Oggetti riferimento o Reference objects si riferiscono al blocco di memoria che gli è stato allocato,
segue che quando un reference object è copiato in una nuova variabile, nel nuovo oggetto si copierà 
il riferimento alla zona di memoria a cui l'oggetto originale si riferisce, evitando di allocare una
nuova zona di memoria. 
\\
Una conseguenza importante del meccanismo dei riferimenti è che qualsiasi cambiamento effettuato su un
oggetto riferimento sarà rispecchiato da tutti gli altri reference objects che possiedono lo stesso riferimento
al blocco di memoria

\newpage
\subsection{Tipi primitivi}
I tipi primitivi sono degli oggetti valore particolari che rappresentano i valori puri del linguaggio come
i numeri, le stringhe di testo, i valori booleani.

\subsubsection{Numeri}
I numeri sono rappresentati dal tipo primitivo "number" che accoglie un qualsiasi numero tale che
sia compreso tra \( \pm5.0*1-^{-324} \) e \(\mp 1.7*10^308\) con una precisione tra le 15 e le 17
cifra, il tipo number equivale al tipo double presente in linguaggi come C, C\#, ecc...

\subsubsection{Stringa}
Il tipo stringa, o string, rappresenta una qualsiasi successione di caratteri codificata in UTF-16, 
non ha una lunghezza massima predefinita.
I valori string sono immutabili (ovvero non è modificabile), quando si eseguono delle espressioni
che andrebbero a modificare il valore di una stringa, viene instanziata una nuova stringa con il valore
modificato.

\subsubsection{Booleano}
Il tipo booleano, o bool, rappresenta una decisione binaria vera o false.
Le variabili di tipo booleano dunque possono assumere due singoli valori rispettivamente 
rappresentati dalle parole chiave "true" per il valore vero e "false" per il valore falso.

\subsubsection{Null}
Il valore null non è un tipo primitivo in se ma serve per indicare l'assenza del valore in una variabile,
Ogni tipo può assumere il valore null.

\newpage
 
\subsection{Espressioni}
Una espressione è una qualsiasi operazione che coinvolge zero o più operatori e uno o più operandi 
(o espressioni) che può essere valutata all’interno del linguaggio. \\
Ogni espressione ha un tipo di ritorno che indica il tipo dell’oggetto che sarà restituito dopo la valutazione 
dell’espressione. \\
Gli operatori sono predefiniti e si possono trovare tutti nella lista di definizione degli operatori. \\
Le espressioni vengono classificate in base al tipo di ritorno o al numero di operatori.
\\
Nella definizione delle espressioni con il tipo del valore si intende una qualsiasi espressione che ritorna 
un oggetto dello stesso tipo.

\subsubsection{Espressioni letterali}
Un’espressione letterale è un qualsiasi valore puro di un tipo primitivo (number, string, bool). \\
Seguono alcuni esempi di espressioni letterali separata dalla virgola: 1, 0.2, “ciao mondo”, true, false

\subsubsection{Espressioni matematiche}
Un'espressione matematica è una qualsiasi operazione che coinvolge gli operatori matematici,
hanno come tipo di ritorno il tipo number
\\
le seguenti sono tutte espressioni matematiche:
\begin{itemize}
    \item Addizione \verb|<numero> + <numero>|
    \item Sottrazione \verb|<numero> - <numero>|
    \item Prodotto \verb|<numero> * <numero>|
    \item Divisione \verb|<numero> / <numero>|
    \item Potenza \verb|<numero> ^ <numero>|
    \item Modulo \verb|<numero> % <numero>|
\end{itemize}
l'ordine di precedenza per la valutazione segue quello matematico:
\begin{enumerate}
    \item Potenza
    \item Prodotto, divisione e Modulo
    \item Addizione e Sottrazione
\end{enumerate}

\subsubsection{Espressioni di confronto}
Le espressioni di confronto coinvolgono gli operatori di maggioranza e minoranza, stretta e larga,
il tipo di ritorno dell'espressione è un booleano il cui valore è il risultato del confronto.
\begin{itemize}
    \item Minore di \verb|<espressione> < <espressione>|
    \item Maggiore di \verb|<espressione> > <espressione>|
    \item Minore uguale di \verb|<espressione> <= <espressione>|
    \item Maggiore uguale di \verb|<espressione> >= <espressione>|
\end{itemize}

\subsubsection{Espressioni di uguaglianza}
Le espressioni di uguaglianza coinvolgono gli operatori di uguaglianza, 
il tipo di ritorno dell'espressione è booleano il cui valore è il risultato dell'uguaglianza. 
\\
Come per le espressioni di confronto, il tipo delle due espressioni operandi deve essere il medesimo
per avere una espressione di uguaglianza legale.
\begin{itemize}
    \item Uguale a \verb|<espressione> == <espression>|
    \item Diverso da \verb|<espressione> != <espression>|
\end{itemize}

\subsubsection{Espressioni logiche condizionali}
Le espressioni logiche condizionali coinvolgono gli operatori logici AND e OR,
il tipo di ritorno dell'espressione è booleano il cui valore è il risultato dell'operazione logica.
\begin{enumerate}
    \item AND \verb|<booleano> && <booleano>|
    \item OR \verb=<booleano> || <booleano>=
\end{enumerate}

Le espressioni logiche condizionali cercano di valutare il minimo possibile per ottimizzare i controlli.
Se una espressione si rivela vera o falsa alla valutazione del primo operando l'operazione non 
valuterà il secondo operando ma ritornerà direttamente il corrispettivo risultato.

\subsubsection{Espressioni logiche binarie}
Le espressioni logiche binarie sono simili alle espressioni logiche condizionali ma valutano sempre
entrambi gli operandi
\begin{enumerate}
    \item AND \verb|<booleano> & <booleano>|
    \item OR \verb=<booleano> | <booleano>=
    \item XOR \verb|<booleano> ^ <booleano>|
    \item NOT \verb|!<booleano>|
\end{enumerate}

\subsubsection{Espressioni di accesso}
Le espressioni di accesso permettono d'interagire con i valori contenuti all'interno delle variabili.
In base all'espressione si accederà a tipologie di oggetti diversi.

\begin{itemize}
    \item 
    {
        Accesso a membro: \verb|<oggetto>.<membro>| utilizza operatore di accesso \verb|.| per accedere
        al valore di uno specifico membro di un oggetto.
        Il tipo di ritorno dell'espressione è il tipo del membro a cui si sta accedendo.
    }
    \item 
    {
        Accesso tramite indice: \verb|<lista>[<numero>]| utilizza l'operatore di accesso tramite indice
        \verb|[<numero>]| per accedere allo specifico indice della lista definito dall'numero all'interno
        dell'operatore, ha come tipo di ritorno lo stesso tipo degli oggetti della lista.
    }
    \item 
    {
        Accesso a variabile: \verb|<identificatore>| l'identificatore unico deve appartenere ad variabile
        per poter essere legale. Accede al valore contenuto nella variabile, il tipo di ritorno dell'espressione
        è lo stesso tipo della variabile a cui si sta accedendo
    }
\end{itemize}

\subsubsection{Espressione tipo}
Le espressioni tipo o type expression, sono tutte quelle espressioni che si occupano della 
gestione dei tipi degli oggetti:
\begin{itemize}
    \item
    {
        Controllo del tipo: \verb|<oggetto> is <tipo>| controlla se l'oggetto è del tipo richiesto.
        L'espressione ha come tipo di ritorno un valore booleano cui valore corrisponde il risultato del controllo:
        vero se l'oggetto è del tipo richiesto, falso altrimenti.
    }
    \item 
    {
        Conversione del tipo, type casting: \verb|<oggetto> as <tipo>| il type casting é utilizzato
        per convertire un oggetto in un oggetto di tipo diverso, il tipo di ritorno é lo stesso del tipo richiesto.
        Nel caso la conversione sia ammissibile l'espressione ristornerà il nuovo oggetto con il tipo prestabilito,
        in caso contrario sarà ritornato un valore nullo.
    }
\end{itemize}

\subsubsection{Espressioni d'istanziazione}
L'espressione d'istanziazione \verb|new <tipo>()| è usata per creare una nuova istanza del tipo
richiesto, il tipo di ritorno corrisponde al tipo richiesto.
\\
L'istanziazione di un oggetto può essere seguita da una lista d'inizializzazione per popolare 
velocemente con dei valori il nuovo oggetto.

\subsubsection{Espressione d'assegnamento}
Sintassi: \verb|<espressione di accesso> = <espressione>;| \\
L'espressione di assegnamento è utilizzata per assegnare il risultato di un espressione
a una variabile, attributo o a uno specifico indice di una lista. \\
Per avere un espressione di assegnamento legale è necessario che i tipi di ritorno
delle due espressioni siano gli stessi. \\
Il tipo di ritorno dell'espressione è nullo.

\subsubsection{Espressione di chiamata}
L'espressione di chiamata è utilizzata per chiamare l'esecuzione di una funzione 
con degli specifici argomenti definiti nella lista degli argomenti.
Il tipo di ritorno dell'espressione corrisponde al tipo di ritorno della funzione.

\subsubsection{Espressione giocatore}
Una espressione giocatore, o player expression, è un espressione speciale utilizzata per ritornare 
un oggetto giocatore. 
\\
Gli oggetti giocatori sono identificati tramite un numero partendo da zero fino al numero di giocatori meno uno,
questi identificatore numerici possono essere utilizzati all'interno delle espressioni giocatore, che equivale 
a una espressione matematica il cui risultato è ristretto per poter essere mappato a un giocatore.
\\
La funzione che si occupa di riportare il risultato dell'espressione matematica nell'intervallo di 
mappatura dei giocatori è la seguente
\[
\begin{cases}
    r \% P  & r >= 0 \\
    (P - |r| \% P) \% P & r < 0
\end{cases}
\] 
Dove:
\begin{itemize}
    \item r: è il risultato dell'equivalente espressione matematica
    \item P: è il numero di giocatori
    \item \%: è l'operatore modulo
\end{itemize}

\subsubsection{Ordine di precedenza}
L'ordine standard (senza modificatori di precedenza) di valutazione delle espressioni è il seguente:
\begin{enumerate}
    \item Espressioni d'istanziazione
    \item Espressioni di chiamata
    \item Espressioni di accesso
    \item Espressioni matematiche
    \item Espressioni giocatore
    \item Espressioni di confronto
    \item Espressioni di uguaglianza 
    \item Espressioni logiche binarie 
    \item Espressioni logiche condizionali 
    \item Espressioni di assegnamento
\end{enumerate}
Se non specificato altrimenti le espressioni all'interno di una stessa categoria 
si risolvo da sinistra verso destra. \\
Qualsiasi espressione può essere inserita all'interno di parentesi tonde per aumentare 
la priorità di valutazione, la valutazione partirà delle espressioni maggiormente innestate.

\newpage
\subsection{Istruzioni}
Le istruzioni, o statements, sono i macro comandi del linguaggio, ovvero una combinazione 
di clausole ed espressioni usate per adempire a un compito preciso.
Le istruzioni devono terminare con il carattere \verb|;|, ma nel caso sia possibile definire
un blocco d'istruzioni associato è possibile omettere il \verb|;|

\subsubsection{Istruzione di ritorno}
Sintassi:   \verb|return [<espressione>];| \\
L’istruzione di ritorno è utilizzata per finire l’esecuzione di un effetto,
ritornando il controllo al codice che ha chiamato la funzione,
restituendo (o no) un risultato in base al tipo di ritorno della funzione. \\
Il risultato restituito proviene dalla valutazione della espressione definita nella istruzione,
nel caso si abbia un tipo di ritorno senza tipo bisogna omettere la espressione.

\subsubsection{Istruzione espressione}
Sintassi: \verb|<espressione>;| \\
l'istruzione espressione, o expression statement, serve per eseguire una certa espressione; 


\subsubsection{Istruzione di dichiarazione}
Sintassi: \verb|<tipo> <identificatore> = <espressione>;| \\
L'istruzione di dichiarazione è utilizzata per definire delle nuove variabili del tipo 
richiesto e inizializzate con il risultato dell'espressione alla destra dell'operatore d'assegnamento. \\
Le variabili dichiarate sono utilizzabili dopo la loro dichiarazione tramite l'identificatore, ovvero
tramite una espressione di accesso. 

\subsubsection{Istruzione di selezione}
Sintassi: \verb|if(<espressione booleana>) <blocco istruzioni>| \\
L'istruzione di selezione è utilizza per selezionare se eseguire o no il blocco d'istruzioni
associato, rispettivamente se la valutazione della espressione booleana abbia esito vero o falso.

\subsubsection{Istruzione Else}
Sintassi \verb|else <blocco istruzioni>|. \\
L'istruzione else permette di eseguire il blocco d'istruzioni associato quando l'espressione
di selezione precedente sia risultata false. \\
Una istruzione else è usabile solo direttamente successivamente a una istruzione di selezione.

\subsubsection{Istruzione Else if}
Sintassi: \verb|else if(<espressione booleana>) <blocco istruzioni>| \\
L'istruzione else if equivale all'unione di un istruzione else e un Istruzione if poiché 
eseguirà il blocco d'istruzioni associato solamente se l'espressione booleana della 
istruzione precedente ha avuto esito falso e se il risultato della espressione booleana 
dell'istruzione else if ha esito vero. \\
Come per l'istruzione else è possibile definirla solo direttamente successivamente a un Istruzione 
di selezione, è possibile concatenare molteplici else if poiché considerata anch'essa come 
istruzione di selezione.

\subsubsection{Istruzione while}
Sintassi: \verb|while(<espressione boolena>) <blocco istruzioni>| \\
L'istruzione while fa parte della categoria delle istruzioni di ciclo, viene usata per ripetere
il blocco d'istruzioni associato fino a che l'espressione booleana, valutata precedentemente a ogni 
esecuzione del blocco, non risulti falsa.

\subsubsection{Istruzione for}
Sintassi: \verb|for([<espressione> | <istruzione di dichiarazione>]; [<espressione booleana>]; [<espressione>]) <blocco istruzioni>| \\
L'istruzione for é un altro modo per creare cicli, utilizzata principalmente come abbreviazione
dell'istruzione while. \\
Si divide in tre parti principali:
\begin{enumerate}
    \item {
        Inizializzazione: è un espressione che viene valutata una volta all'inizio del ciclo, \\
        é legale definire al posto di un espressione una istruzione di dichiarazione
        (la variabile dichiarata avrà come ambito il blocco dell'istruzione for).
    }
    \item { 
        Espressione di controllo: definisce se continuare a eseguire il ciclo se risulta vera 
        o terminarlo se risulta falsa. \\
        Se omessa verrà preso come default l'espressione letterale \verb|true| creando un ciclo infinito.
    }
    \item {
        Espressione post ciclo: è un espressione che viene valutata dopo l'esecuzione del blocco di istruzinoi 
        del ciclo ma prima della valutazione della espressione di controllo.
    }
\end{enumerate}

\subsubsection{Istruzione continue}
Sintassi: \verb|continue;|
L'istruzione continue avvia una nuova iterazione dell'istruzione di ciclo più vicina che la racchiude.

\subsubsection{Istruzione break}
Sintassi: \verb|break;|
L'istruzione break termina l'istruzione di ciclo più vicina che la racchiude.

\section{Classi}
Sintassi: \verb|<visibilità> class <identificatore> <blocco classe>|
Una classe é utilizzata per definire un insieme di dati correlati, detti attributi, e di 
comportamenti che li modificano.

All'interno del blocco associato, detto blocco della classe o class block sono definibili
solamente i seguenti tag:
\begin{itemize}
    \item attribute
    \item function
    \item action
\end{itemize}
Nel blocco è possibile riferirsi all'istanza stessa dell'oggetto tramite 
la parola chiave \verb|this|.

\subsection{Attributi}
Un attributo è una variabile associata alla class in cui é dichiarato, il valore in se della 
variabile é legato all'oggetto ovvero all'istanza della classe. \\
Per definire un attribute si utilizza il tag \verb|attribute| come nella seguente sintassi:
\begin{verbatim}
attribute <identificatore>
{
    returns <tipo>;     // definisce il tipo dell'attributo
    default {           // definisce il valore di prima inizializzazione dell'attributo
        returns <expressione di tipo>;
    }
}
\end{verbatim}
Essendo una sintassi prolissa, si offre un alternativa sintattica tramite l'utilizza
delle istruzioni di dichiarazione e inizializzazione.
\begin{verbatim}
<tipo> <identificatore> = <default>; 
\end{verbatim}

\subsection{Visibilità}
La visibilità di una classe definisce il suo comportamento e l'assegnamento rispetto al giocatore.

\subsubsection{Visibilità locale}
La visibilità locale o \verb|local| è considerata il default, ovvero se utilizzata se non è specificato
altrimenti, possono esistere molteplici istanze di una classe locale create tramite l'espressione di istanziazione. \\
In generale le classi locale servono per definire i ruoli, con le loro azioni, che i giocatori possono
intraprendere nel corso della partita. \\
Quando si assegna un'Istanza di una classe locale a un giocatore, il giocatore precedente perde la proprietà dell'istanza
concedendola al giocatore assegnato.

\subsubsection{Visibilità di gruppo}
La visibilità di gruppo o \verb|group| è utilizzata per specificare delle classi le cui istanze possono
essere condivise tra molteplici giocatori, cui possono accedere a le azioni fornite dalla classe come 
se fosse di loro proprietà. \\
Come  per le classi a visibilità locale si possono creare molteplici istanze tramite l'espressione di istanziazione. \\
Il principale utilizzo di una classe di gruppo é per creare dei gruppi di giocatori che abbiano
azioni e attributi comuni. \\
Quando si assegna un giocatore a una istanza di una classe di gruppo non avviene nessun trasferimento
di proprietà, ma si aggiunge all'insieme di giocatori che la condividono.

\subsubsection{Visibilità globale}
La visibilità globale o \verb|global| è utilizzata per specificare delle classi che sono 
limitate a una sola istanza in tutto il programma. \\
Tutti i giocatori possono accedere alle azioni delle classi globali.

\subsection{Azioni}
Le azioni rappresentano cosa possono fare i giocatori durante il corso del gioco, possono essere avviate
automaticamente all'avvenire di certi eventi oppure invocate tramite un espressione di chiamata.\\
Un'azione é definita all'interno di una classe tramite il relativo tag \verb|action <identificatore> <blocco azione>|.

\subsubsection{Argomenti dell'azione}
\begin{verbatim}
[optional] [<modificatore>] input <tipo> <identificatore> [<filtro>]
\end{verbatim}
Gli argomenti o inputs di un’azione sono dei tag speciali che servono per definire delle variabili 
inizializzate in base alla tipologia dell’argomenti e utilizzate all’interno dei 
blocchi effetto di un’azione. \\
La tipolgoia dell'input é definita dal suo modificatore che ne influenza il comportamento, 
i due principali sono:

\paragraph{Modificatore giocatore}
Sintassi: \verb|player [<espressione giocatore>]| \\
Detto anche modificatore player, è il default in caso di omissione del modificatore nella
dichiarazione di un input. \\
Un input con il modificatore giocatore chiede direttamente al giocatore bersaglio di 
selezionare i valori con cui popolare l'input. \\
Se non specificato, il giocatore bersaglio di default dipende dalla visibilità della classe:
\begin{itemize}
    \item Locale: il giocatore bersaglio è il proprietario della classe
    \item Globale: il giocatore bersaglio é il giocatore attivo 
    \item {
        Gruppo, si hanno due casi:
        \begin{itemize}
            \item Se il giocatore attivo fa parte del gruppo sará utilizzato come giocatore bersaglio
            \item Se il giocatore attivo non fa parte del gruppo, un giocatore a caso sará utilizzato come giocatore bersaglio
        \end{itemize}
    }
\end{itemize}

\paragraph{Modificatore auto}
Il modificatore auto o automatico non necessita di nessun input, il programma ha la responsabilitá di
filtrare tutte le istanzi accettabili popolando la variabile definita dall'input. \\
Possono presentarsi dei fallimenti inattesi nel caso l'input non sia opzionale,
non si tratti di una lista e nessuna istanza soddisfi i filtri.

\paragraph{Opzionale}
La parola chiave opzionale é un modificatore opzionale che permette alla variabile definita dall'input
di assumere valore nullo quando l'input inizializza la variabile. \\ 
Gli input non opzionali garantiscono che forniranno dei valori non nulli oppure l'azione non sarà eseguita

\subsubsection{Filtri degli input}

\subsubsection{Effetti}
\subsubsection{Requisiti}
\subsubsection{Triggers}

\section{Eventi}

\section{Funzioni}

\section{Interactables}

\section{Mazzi}

\section{Giocatori}

\section{Boards}

\section{Ereditarietà}





