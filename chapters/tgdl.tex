\section{Nozioni Fondamentali}

\subsection{Identificatori}
Gli Identificatori sono una qualsiasi stringa di caratteri che inizia per un trattino basso "\_" o 
per lettera, opzionalmente seguiti da un qualsiasi ammontare di lettere o numeri senza spazi.

\subsection{Blocchi}
I Blocchi sono un insieme struttura d'istruzioni, tag e altri blocchi.
Un blocco si definisce come tutto il contenuto all'interno di una coppia di parentesi graffe,
rispettivamente aperta e chiusa. 
\\
Il blocco definire l'ambito, o scope, e vita di una variabile definita al suo interno ovvero
il suo campo d'azione che parte dall riga dov'è stata dichiarata fino alla fine del blocco 
che contiene la sua dichiarazione.
\\
La Visibilità di una variabile si estende a tutti gli ambiti figli del blocco di dichiarazione, 
dove figlio significa definito all'interno del padre (annidato).

 
\subsection{Tags}
I Tag sono delle parole chiave riservate che modificano il comportamento del blocco associato, 
ogni tag descrive cosa è permesso definire al suo interno.
All'interno della proposta è possibile che si trovi la dicitura "blocco <tag>" come abbreviazione
per indicare la coppia tag e relativo blocco associato.

 
\subsection{Commenti}
I commenti sono linee di codice che vengono ignorare durante l'esecuzione del programma, 
per commentare una riga è sufficiente inserire una doppia barra "//", qualsiasi scritta oltre 
le due barre è considerata parte del commento.

\section{Primitive del linguaggio}
Tabletop Game Description Language si basa su un insieme di primitive da cui sono costruire tutte 
le funzionalità del linguaggio.

 
\subsection{Oggetti}
Un oggetto è un'istanza di una certa classe, un blocco di memoria che viene allocato e configurato secondo
le specifiche definite dalla classe. \\
Gli oggetti hanno due comportamenti principali, gli oggetti valori e gli oggetti riferimento:
\\
Oggetti valore o Value objects, si riferiscono direttamente al valore, quando avviene una copia dell'oggetto
viene creata una nuova istanza della stessa tipologia di oggetto che contiene lo stesso valore.
Il comportamento degli oggetti valore è principalmente per i tipi primitivi come number, string, bool.
\\
Oggetti riferimento o Reference objects si riferiscono al blocco di memoria che gli è stato allocato,
segue che quando un reference object è copiato in una nuova variabile, nel nuovo oggetto si copierà 
il riferimento alla zona di memoria a cui l'oggetto originale si riferisce, evitando di allocare una
nuova zona di memoria. 
\\
Una conseguenza importante del meccanismo dei riferimenti è che qualsiasi cambiamento effettuato su un
oggetto riferimento sarà rispecchiato da tutti gli altri reference objects che possiedono lo stesso riferimento
al blocco di memoria

\newpage
\subsection{Tipi primitivi}
I tipi primitivi sono degli oggetti valore particolari che rappresentano i valori puri del linguaggio come
i numeri, le stringhe di testo, i valori booleani.

\subsubsection{Numeri}
I numeri sono rappresentati dal tipo primitivo "number" che accoglie un qualsiasi numero tale che
sia compreso tra \( \pm5.0*1-^{-324} \) e \(\mp 1.7*10^308\) con una precisione tra le 15 e le 17
cifra, il tipo number equivale al tipo double presente in linguaggi come C, C\#, ecc...

\subsubsection{Stringa}
Il tipo stringa, o string, rappresenta una qualsiasi successione di caratteri codificata in UTF-16, 
non ha una lunghezza massima predefinita.
I valori string sono immutabili (ovvero non è modificabile), quando si eseguono delle espressioni
che andrebbero a modificare il valore di una stringa, viene instanziata una nuova stringa con il valore
modificato.

\subsubsection{Booleano}
Il tipo booleano, o bool, rappresenta una decisione binaria vera o false.
Le variabili di tipo booleano dunque possono assumere due singoli valori rispettivamente 
rappresentati dalle parole chiave "true" per il valore vero e "false" per il valore falso.

\subsubsection{Null}
Il valore null non è un tipo primitivo in se ma serve per indicare l'assenza del valore in una variabile,
Ogni tipo può assumere il valore null.

\newpage
 
\subsection{Espressioni}
Una espressione è una qualsiasi operazione che coinvolge zero o più operatori e uno o più operandi 
(o espressioni) che può essere valutata all’interno del linguaggio. \\
Ogni espressione ha un tipo di ritorno che indica il tipo dell’oggetto che sarà restituito dopo la valutazione 
dell’espressione. \\
Gli operatori sono predefiniti e si possono trovare tutti nella lista di definizione degli operatori. \\
Le espressioni vengono classificate in base al tipo di ritorno o al numero di operatori.
\\
Nella definizione delle espressioni con il tipo del valore si intende una qualsiasi espressione che ritorna 
un oggetto dello stesso tipo.

\subsubsection{Espressioni letterali}
Un’espressione letterale è un qualsiasi valore puro di un tipo primitivo (number, string, bool). \\
Seguono alcuni esempi di espressioni letterali separata dalla virgola: 1, 0.2, “ciao mondo”, true, false

\subsubsection{Espressioni matematiche}
Un'espressione matematica è una qualsiasi operazione che coinvolge gli operatori matematici,
hanno come tipo di ritorno il tipo number
\\
le seguenti sono tutte espressioni matematiche:
\begin{itemize}
    \item Addizione \verb|<numero> + <numero>|
    \item Sottrazione \verb|<numero> - <numero>|
    \item Prodotto \verb|<numero> * <numero>|
    \item Divisione \verb|<numero> / <numero>|
    \item Potenza \verb|<numero> ^ <numero>|
    \item Modulo \verb|<numero> % <numero>|
\end{itemize}
l'ordine di precedenza per la valutazione segue quello matematico:
\begin{enumerate}
    \item Potenza
    \item Prodotto, divisione e Modulo
    \item Addizione e Sottrazione
\end{enumerate}

\subsubsection{Espressioni di confronto}
Le espressioni di confronto coinvolgono gli operatori di maggioranza e minoranza, stretta e larga,
il tipo di ritorno dell'espressione è un booleano il cui valore è il risultato del confronto.
\begin{itemize}
    \item Minore di \verb|<espressione> < <espressione>|
    \item Maggiore di \verb|<espressione> > <espressione>|
    \item Minore uguale di \verb|<espressione> <= <espressione>|
    \item Maggiore uguale di \verb|<espressione> >= <espressione>|
\end{itemize}

\subsubsection{Espressioni di uguaglianza}
Le espressioni di uguaglianza coinvolgono gli operatori di uguaglianza, 
il tipo di ritorno dell'espressione è booleano il cui valore è il risultato dell'uguaglianza. 
\\
Come per le espressioni di confronto, il tipo delle due espressioni operandi deve essere il medesimo
per avere una espressione di uguaglianza legale.
\begin{itemize}
    \item Uguale a \verb|<espressione> == <espression>|
    \item Diverso da \verb|<espressione> != <espression>|
\end{itemize}

\subsubsection{Espressioni logiche condizionali}
Le espressioni logiche condizionali coinvolgono gli operatori logici AND e OR,
il tipo di ritorno dell'espressione è booleano il cui valore è il risultato dell'operazione logica.
\begin{enumerate}
    \item AND \verb|<booleano> && <booleano>|
    \item OR \verb|<booleano> || <booleano>|
\end{enumerate}

Le espressioni logiche condizionali cercano di valutare il minimo possibile per ottimizzare i controlli.
Se una espressione si rivela vera o falsa alla valutazione del primo operando l'operazione non 
valuterà il secondo operando ma ritornerà direttamente il corrispettivo risultato.

\subsubsection{Espressioni logiche binarie}
Le espressioni logiche binarie sono simili alle espressioni logiche condizionali ma valutano sempre
entrambi gli operandi
\begin{enumerate}
    \item AND \verb|<booleano> & <booleano>|
    \item OR \verb|<booleano> | <booleano>|
    \item XOR \verb|<booleano> ^ <booleano>|
    \item NOT \verb|!<booleano>|
\end{enumerate}

\subsubsection{Espressioni di accesso}
Le espressioni di accesso permettono d'interagire con i valori contenuti all'interno delle variabili.
In base all'espressione si accederà a tipologie di oggetti diversi.

\begin{itemize}
    \item 
    {
        Accesso a membro: \verb|<oggetto>.<membro>| utilizza operatore di accesso \verb|.| per accedere
        al valore di uno specifico membro di un oggetto.
        Il tipo di ritorno dell'espressione è il tipo del membro a cui si sta accedendo.
    }
    \item 
    {
        Accesso tramite indice: \verb|<lista>[<numero>]| utilizza l'operatore di accesso tramite indice
        \verb|[<numero>]| per accedere allo specifico indice della lista definito dall'numero all'interno
        dell'operatore, ha come tipo di ritorno lo stesso tipo degli oggetti della lista.
    }
    \item 
    {
        Accesso a variabile: \verb|<identificatore>| l'identificatore unico deve appartenere ad variabile
        per poter essere legale. Accede al valore contenuto nella variabile, il tipo di ritorno dell'espressione
        è lo stesso tipo della variabile a cui si sta accedendo
    }
\end{itemize}

\subsubsection{Espressione tipo}
Le espressioni tipo o type expression, sono tutte quelle espressioni che si occupano della 
gestione dei tipi degli oggetti:
\begin{itemize}
    \item
    {
        Controllo del tipo: \verb|<oggetto> is <tipo>| controlla se l'oggetto è del tipo richiesto.
        L'espressione ha come tipo di ritorno un valore booleano cui valore corrisponde il risultato del controllo:
        vero se l'oggetto è del tipo richiesto, falso altrimenti.
    }
    \item 
    {
        Conversione del tipo, type casting: \verb|<oggetto> as <tipo>| il type casting é utilizzato
        per convertire un oggetto in un oggetto di tipo diverso, il tipo di ritorno é lo stesso del tipo richiesto.
        Nel caso la conversione sia ammissibile l'espressione ristornerà il nuovo oggetto con il tipo prestabilito,
        in caso contrario sarà ritornato un valore nullo.
    }
\end{itemize}

\subsubsection{Espressioni d'istanziazione}
L'espressione d'istanziazione \verb|new <tipo>()| è usata per creare una nuova istanza del tipo
richiesto, il tipo di ritorno corrisponde al tipo richiesto.
\\
L'istanziazione di un oggetto può essere seguita da una lista d'inizializzazione per popolare 
velocemente con dei valori il nuovo oggetto.

\subsubsection{Espressione di chiamata}
L'espressione di chiamata è utilizzata per chiamare l'esecuzione di una funzione 
con degli specifici argomenti definiti nella lista degli argomenti.
Il tipo di ritorno dell'espressione corrisponde al tipo di ritorno della funzione.

\subsubsection{Espressione giocatore}
Una espressione giocatore, o player expression, è un espressione speciale utilizzata per ritornare 
un oggetto giocatore. 
\\
Gli oggetti giocatori sono identificati tramite un numero partendo da zero fino al numero di giocatori meno uno,
questi identificatore numerici possono essere utilizzati all'interno delle espressioni giocatore, che equivale 
a una espressione matematica il cui risultato è ristretto per poter essere mappato a un giocatore.
\\
La funzione che si occupa di riportare il risultato dell'espressione matematica nell'intervallo di 
mappatura dei giocatori è la seguente
\[
\begin{cases}
    r \% P  & r >= 0 \\
    (P = |r| \% P) \% P & r < 0
\end{cases}
\] 
Dove:
\begin{itemize}
    \item r: è il risultato dell'equivalente espressione matematica
    \item P: è il numero di giocatori
    \item \%: è l'operatore modulo
\end{itemize}


\newpage
\subsection{Istruzioni}
Tutte le operazioni matematiche sono scrivibili in maniera contratta se si assegna 
vuole assegnare il loro risultato a una variabile mettendo l’operatore come 
prefisso all’operatore di assegnamento (=).
\newpage
 
\section{Classi}

 
\subsection{Attributi}

\subsection{Visibilità}

\subsection{Azioni}

\section{Eventi}

\section{Funzioni}

\section{Interactables}

\section{Mazzi}

\section{Giocatori}

\section{Boards}

\section{Ereditarietà}





