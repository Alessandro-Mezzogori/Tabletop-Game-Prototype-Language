\textit{Tabletop Game Prototype Language} fornisce molti degli strumenti necessari 
a una protipizzazione agile, distanziandosi dal concetto di un gioco come macchina 
a stati finiti, riconducendo lo sviluppo a un architettura a eventi, ovvero 
a una visione azione e reazione. \\
Durante la protipizzazione ne consegue una incrementata abilità comunicativa tra 
gli sviluppatori, progettisti e possibili esterni, grazie a una sintassi maggiormente
esplicitativa a discapito di diventare prolissa. \\
Nonostante la maggior chiarezza, presenta dei punti di espansione non triviali:
\begin{itemize}
    \item Gestione della concomitanza delle azioni di molteplici giocatori.
    \item { 
        Un sistema di gestione del tabellone meno ingrombante nel programma e
        maggiormente permissiva.
    }
    \item {
        Gestione delle espansioni, ovvero aggiunte o modifica a sistemi stabiliti
        nel gioco principale.
    }
\end{itemize}
Inoltre, rispetto alla prototipizzazione in versione cartacea presenta svantaggi
e vantaggi, principalmente il tempo d'implementazione di una nuova funzionalitá o
modifica di una esistente é piú lenta, ma in opposizione permette 
una condivisione piú efficiente tramite i vari sistemi di condivisione digitale.
