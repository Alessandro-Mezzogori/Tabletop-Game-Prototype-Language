\section{Cos'è} % TODO titolo migliore
Parte del "General Game Playing Project" dell'università di Stanford,
è stata concepita da Michael Genesereth come una rappresentazione di alto livello per 
il formalismo di descrivere le regole di qualsiasi gioco. \\
Basato sulla concenzione dei giochi da tavolo come macchine a stati finiti in cui il singolo stato
è un record di un database,
permette di condividere un singolo agente, con la conoscenza logica della sintassi di gdl,
attraverso molteplici possibili giochi; al contrario dei modelli super specializzati in uno solo o 
in un piccolo insieme come DeepMind AlphaStar \cite{AlphaStar},
che ha superato i limiti umani nel famoso videogioco di strategia in tempo reale "StarCraft 2",
o DeepMind AlphaZero \cite{AlphaZero} che conquistò scacchi, shogi e Go. \\

\section{Sintassi}
La sintassi originale di Game Description Language è molto semplice, composta da solo nove parole chiave:
\begin{center}
\begin{tabular}{|c|p{5cm}|}
    \hline
    role(r) & definisce un ruolo r \\
    \hline
    base(p) & definisce una proposizione base p \\
    \hline
    action(a) & definisce un azione possibile a \\
    \hline
    init(p) & definisce che la proposizione p é vera nello stato iniziale del gioco \\ 
    \hline
    legal(a) & definisce il giocatore in controllo puó legalmente eseguire l'azione a \\
    \hline
    goal(r, n) & definisce che lo stato corrente ha punteggio n per il giocatore r \\
    \hline
    terminal & definisce che lo stato corrente é uno stato finale del gioco \\ 
    \hline
\end{tabular}
\end{center}
Inoltre per avere un programma legale, gdl richiede una definizione precisa di role, base, action, init;
le ultime tre definite attraverso l'operatore di definizione é \lstinline|:-|. \\
\begin{lstlisting}[caption=Esempio: definizione cella di tris]
base(cell(M,N,x)) :- index(M) & index(N)
base(cell(M,N,o)) :- index(M) & index(N)
base(cell(M,N,b)) :- index(M) & index(N)
\end{lstlisting}

% continue da link http://ggp.stanford.edu/chapters/chapter_02.html

\subsection{Estensioni}
% Menziona le estensioni GDL 2 e GDL 3 e rispettivamente cosa aggiungono 

\section{Utilizzi principali}
% Machine learning ( esempi di utilizzo in machine learnign ? )
% General game playing competition


\section{Distanza dal modello mentale dei giochi}
Il suo utilità nel ambito di descrivere i giochi come cambiamenti di stati 
utile per intelligenza artificiale, in ambito umano a meno che uno non sia 
abituato lo rende più complesso da capire. 

\begin{itemize}
    \item Difficile da capire un gioco come sequenza di stati legali
    \item Difficile capire significa anche da scrivere
    \item inadatto per creare dei prototipi veloci per testare 
\end{itemize}