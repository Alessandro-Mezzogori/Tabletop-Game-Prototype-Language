\textit{Game Description Language} é un formalismo di alto livello per descrivere le meccaniche 
e regole di un generico gioco,
basato sulla visione di quest'ultimo come una macchina a stati finiti. \\
È stato proposto da Michael Genesereth \cite{GDLSpecification}, professore nel dipartimento d'informatica
all'università di Stanford, come parte del \textit{General Game Playing Project}.
Il progetto ha l'obiettivo di sviluppare degli agenti generici che possano interagire 
con un gioco sconosciuto senza nessuno intervento umano. \\
Al contrario dei modelli specializzati in uno o in un piccolo insieme di giochi come 
DeepMind \textit{AlphaZero} \cite{AlphaZero} che conquistò scacchi, 
shogi e Go battendo anche i rispettivi migliori giocatori al mondo, sia umani sia digitali. \\
La flessibilitá dei giocatori generali apre la possibilitá di creare dei modelli d'intelligenza artificiale 
che possano dedurre dalla sola descrizione logica un metodo efficacie per risolvere qualsiasi tipologia di gioco,
sia complessa come scacchi e go, ma anche semplice come tris. \\
\textit{GDL} permette agli agenti GGP (\textit{General Game Playing Agents}) di avere un' 
unica sintassi capace di descrivere un qualsiasi ambiente di gioco, consentendo 
ai sudetti agenti di estrapolare le relazioni tra i vari nodi di sintassi del linguaggio,
imparando i concetti di alto livello necessari per una prestazione generale.

\section{Sintassi}
Esistono quattro classi di simboli: le costanti di oggetto, 
le costanti di funzione, le costanti di relazione e le variabili. \\
\begin{itemize}
    \item Termine: é o una costante di oggetto o una variabile o un termine funzionale.
    \item Atomo: é un espressione formata dalla relazione tra una costante e n termini
    \item Letterale: un atomo o la negazione di un atomo
    \item {
        Regola: un espressione composta da una testa, l'operator \lstinline|:-| e una congiunzione, 
        marcata dall'operatore \lstinline|&|, di zero o piú letterali.
    }
\end{itemize}
\begin{center}
\begin{tabular}{|c|p{10cm}|}
    \hline
    role(r) & definisce un ruolo \lstinline|r| nel gioco \\
    \hline
    base(p) & definisce una proposizione base \lstinline|p| del gioco \\
    \hline
    input(r, a) & definisce \lstinline|a| come un azione del ruolo \lstinline|r| \\
    \hline
    init(p) & definisce la proposizione \lstinline|p| come vera nello stato iniziale del gioco \\ 
    \hline
    true(p) & definisce la proposizione \lstinline|p| come vera nello stato corrente del gioco \\
    \hline
    does(r, a) & definisce che il giocatore \lstinline|r| esegue l'azione \lstinline|a| \\
    \hline
    next(p) & definisce la proposizione \lstinline|p| come vera nello stato successivo \\
    \hline
    legal(r, a) & definisce che é legale per il ruolo \lstinline|r| eseguire l'azione \lstinline|a| \\
    \hline
    goal(r, n) & definisce che lo stato corrente ha utilità \lstinline|n| per il giocatore \lstinline|r| \\
    \hline
    terminale & definisce lo stato corrente come uno stato terminale \\ 
    \hline
\end{tabular}
\end{center}
Inoltre per avere un programma legale si hanno le seguenti regole:
\begin{itemize}
    \item Si richiede una definizione precisa di role, base, action e init
    \item Bisogna definire legal, goal e terminal, in funzione di una relazione true 
    \item Bisogna definire next in termini di relazioni true e does
    \item Bisogna che non ci siano regole con nessuna relazione true o does nella testa
\end{itemize} 
\begin{lstlisting}[caption=Esempio: definizione cella di tris]
base(cell(M,N,x)) :- index(M) & index(N)
base(cell(M,N,o)) :- index(M) & index(N)
base(cell(M,N,b)) :- index(M) & index(N)
\end{lstlisting}
Oltre alle primitive di gdl é possibile definire delle proprie relazioni come nell'esempio seguente:
\begin{lstlisting}[caption=colonnet e righe in tris]
line(X) :- row(M,X)
line(X) :- column(M,X)

row(M,X) :- 
    true(cell(M,1,X)) &
    true(cell(M,2,X)) &
    true(cell(M,3,X)) 

column(M,X) :- 
    true(cell(1,M,X)) &
    true(cell(2,M,X)) &
    true(cell(3,M,X)) 
\end{lstlisting} 


\subsection{Estensioni}
La sintassi originale di Game Description Language venne poi estesa da Michael Thielscher per le descrizioni di giochi 
a informazione incompleta \cite*{GDL2} e per aggiungere l'introspezione al linguaggio  \cite*{GDL3}

\section{Distanza dal giocatore umano}
La semplicitá e forza nell'ambito del General Game Playing della Game Description Language e affini, 
uniti alla rapida espansione del machine learning ha creato un monopolio sui linguaggi di descrizione
logica dei giochi, creando una carenza di uno specializzato nel descrivere i giochi dal
lato della prototipizzazione, ricerca e utilizzo del prototipo tramite giocatore umano. \\ 
Le principali carenze di GDL come linguaggio di protipizzazione sono:
\begin{itemize}
    \item { 
        é difficoltoso creare un gioco tramite la sua rappresentazione come macchina a stati finiti, 
    }
    \item {
        un gioco in stato embrionale non avrá i concetti base ben delimitati, segue che non sará 
        possibile definire le primitive del gioco in maniera efficacie.
    }
    \item {
        la mancanza di strutture e funzionalità base che sono condivise con la maggior parte dei         
        giochi: carte, tabelloni, mazzi, ect...
    }
\end{itemize}
Nel prossimo capitolo si esplorerá una proposta a un linguaggio di prototipizzazione, che 
tenta di risolvere le problematiche identificate:
\begin{itemize}
    \item Rappresentazione del gioco tramite eventi e risposte a eventi.
    \item Senza bisogno di concetti ben definiti da utilizzare come fondamenta.
    \item Fornisce alcune implementazioni base di funzionalità e strutture dati come liste, stack, mazzi, ecc... 
\end{itemize}