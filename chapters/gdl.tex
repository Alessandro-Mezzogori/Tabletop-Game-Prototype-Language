\section{Cos'è} % TODO titolo migliore
Parte del "General Game Playing Project" dell'università di Stanford,
è stata concepita da Michael Genesereth come una rappresentazione di alto livello per 
il formalismo di descrivere le regole di qualsiasi gioco. \\
Basato sulla concenzione dei giochi da tavolo come macchine a stati finiti in cui il singolo stato
è un record di un database,
permette di condividere un singolo agente, con la conoscenza logica della sintassi di gdl,
attraverso molteplici possibili giochi; al contrario dei modelli super specializzati in uno solo o 
in un piccolo insieme come DeepMind AlphaStar \cite{AlphaStar},
che ha superato i limiti umani nel famoso videogioco di strategia in tempo reale "StarCraft 2",
o DeepMind AlphaZero \cite{AlphaZero} che conquistò scacchi, shogi e Go. \\
La flessibilitá dei giocatori generali apre la possibilitá di creare dei modelli d'intelligenza artificiale 
che possa dedurre dalla sola descrizione logica un metodo efficacie per risolvere qualsiasi tipologia di gioco,
sia complessa come scacchi e go ma anche semplice come tris.  

\section{Sintassi}
Esistono quattro classi di simboli: le costanti di oggetto, 
le costanti di funzione, le costanti di relazione e le variabili. \\
\begin{itemize}
    \item Termine: é o una costante di oggetto o una variabile o un termine funzionale.
    \item Atomo: é un espressione formata dalla relazione tra una costante e n termini
    \item Letterale: un atomo o la negazione di un atomo
    \item {
        Regola: un espressione composta da una testa, l'operator \lstinline|:-| e una congiunzione, 
        marcata dall'operatore \lstinline|&|, di zero o piú letterali.
    }
\end{itemize}
\begin{center}
\begin{tabular}{|c|p{5cm}|}
    \hline
    role(r) & definisce un ruolo \lstinline|r| nel gioco \\
    \hline
    base(p) & definisce una proposizione base \lstinline|p| del gioco \\
    \hline
    input(r, a) & definisce \lstinline|a| come un azione del ruolo \lstinline|r| \\
    \hline
    init(p) & definisce la proposizione \lstinline|p| come vera nello stato iniziale del gioco \\ 
    \hline
    true(p) & definisce la proposizione \lstinline|p| come vera nello stato corrente del gioco \\
    \hline
    does(r, a) & definisce che il giocatore \lstinline|r| esegue l'azione \lstinline|a| \\
    \hline
    next(p) & definisce la proposizion \lstinline|p| come vera nello stato successivo \\
    \hline
    legal(r, a) & definisce che é legale per il ruolo \lstinline|r| eseguire l'azione \lstinline|a| \\
    \hline
    goal(r, n) & definisce che lo stato corrente ha utilità \lstinline|n| per il giocatore \lstinline|r| \\
    \hline
    terminale & definisce lo stato corrente come uno stato terminale \\ 
    \hline
\end{tabular}
\end{center}
Inoltre per avere un programma legale si hanno le seguenti regole:
\begin{itemize}
    \item Si richiede una definizione precisa di role, base, action e init
    \item Bisogna definire legal, goal e terminal in funzione di una relazione true 
    \item Bisogna definire next in termini di relazioni true e does
    \item Bisogna che non ci siano regole con nessuna di relazione true o does nella testa
\end{itemize} 
\begin{lstlisting}[caption=Esempio: definizione cella di tris]
base(cell(M,N,x)) :- index(M) & index(N)
base(cell(M,N,o)) :- index(M) & index(N)
base(cell(M,N,b)) :- index(M) & index(N)
\end{lstlisting}
Oltre alle primitive di gdl é possibile definire delle proprie relazioni come nell'esempio seguente:
\begin{lstlisting}[caption=colonnet e righe in tris]
line(X) :- row(M,X)
line(X) :- column(M,X)

row(M,X) :- 
    true(cell(M,1,X)) &
    true(cell(M,2,X)) &
    true(cell(M,3,X)) 

column(M,X) :- 
    true(cell(1,M,X)) &
    true(cell(2,M,X)) &
    true(cell(3,M,X)) 
\end{lstlisting} 


\subsection{Estensioni}
La sintassi originale di Game Description Language venne poi estesa da Michael Thielscher per le descrizioni di giochi 
a informazione incompleta \cite*{GDL2} e per aggiungere l'introspezione al linguaggio  \cite*{GDL3}

\section{Distanza dal giocatore umano}
Se un giocatore generale artificiale beneficia enormemente dalla descrizione logica di un gioco, 
e quindi dalla sua interpretazione come macchina a stati finiti, un umano al contrario non razionalizza
un gioco da tavolo come una serie di stati ma piú come 

\begin{itemize}
    \item mancanza di un linguaggio per protipizzare giochi interagenti con umani
    \item Difficile da capire un gioco come sequenza di stati legali
    \item Difficile capire significa anche da scrivere
    \item inadatto per creare dei prototipi veloci per testare 
\end{itemize}